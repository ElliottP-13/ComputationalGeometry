\documentclass[11pt]{article}

% set these commands
\newcommand{\course}{CSCI 534}
\newcommand{\proj}{Final Project}
\newcommand{\instructor}{David L. Millman}

\usepackage{macros}

\begin{document}

{ ~\\
    \course \\ 
    \proj \\ 
    \instructor
    \vspace{1em}
}

All students (even auditors) must participate in a final project.  You may work
in groups with one to four members. Larger groups are allowed with permission.
There are two tracks, you must pick one (see below).  I am happy to participate
in group discussion (even if it is outside of office hours).  We will still have
homework assignments, however, the assignments will be substantially easier to
provide you with adequate time to work on the final project.

On April 6th, in class, each team will have a few minutes to describe what they
are working on. Teams can morph and change until the end of the day April 8th.

\paragraph{Goal for paper track} Explore an open research problem and write a
short paper.  The paper should include a problem definition and partial results.
Expected length is about 6 pages.

There are a lot of open problem lists in Computational Geometry around the
internet.  For example The Open Problems Project
(\url{https://cs.smith.edu/~jorourke/TOPP/}). Also feel free to come and speak
with me to get additional ideas.


\paragraph{Goal for implementation track} Do an implementation project and a
brief report.  You have a lot of freedom.  The project can be from class or your
own research interest (but still related to Computational Geometry).  You can
animate and algorithm or prototype and implementation idea.  Again, feel free
to come speak with me to get additional ideas.

\paragraph{Deliverables}

\begin{itemize}

    \item now-Fri April 2: Send a direct message on slack or speak with me about
        a problem that you are interested in so that I can point out resources
        and make sure you are not working on something that is too hard or
        already solved.

    \item April 6: In class, present problem description (at most 5 min).

    \item April 8: On slack, direct message me a one page outline and plan of
        attack.

    \item April 15: On the course's slack channel post a progress report.

    \item April 22: First draft (for paper track) / progress report
        (implementation track)

    \item April 27: Final version

    \item April 27+29: 10-15 min presentations

    \item May 7 (optional): Canadian Conference in Computational Geometry
        (CCCG) takes short papers. This summer, CCCG is in Halifix Conference
        site: \url{https://projects.cs.dal.ca/cccg2021/}.
\end{itemize}

\paragraph{Acknowledgements} Project based from Jack Snoeyink's CG course.

\end{document}
