\documentclass[11pt]{article}

% set these commands
\newcommand{\course}{CSCI 534}
\newcommand{\proj}{Homework 06}
\newcommand{\dueDate}{2021-04-22}
\newcommand{\instructor}{David L. Millman}

\usepackage{macros}

\newcommand{\A}{{\mathcal{A}}}

\begin{document}

{ ~\\
    \course \\ 
    \proj \\ 
    Due \dueDate \\
    \instructor
}



\vspace{1em}

This homework assignment occurs at the same time as the course project.  These
questions should take substantially less time then the first five assignments.
While I invite you to spend as much time on these problems as you like, I do not
expect you to spend any more than 3 hours on the entire assignment so that you
can focus on your projects.

\section*{Problem 1 (10 Points)}

We saw in class that the Voronoi diagram of a set of points in $\R^2$ is the
projection of the upper envelope of the dual lifted set of planes in $\R^3$.
What does the projection of the lower envelope correspond to? Similarly, what
does the projection of the upper convex hull of the points lifted to $\R^3$
correspond to?

You MAY (but do not need to) answer these questions by researching on the
internet. Cite the sources you were using and give an explanation in your own
words

\section*{Problem 2 (10 Points)}

We saw in class how to solve motion planning problems in $\R^2$ in a static
environment of polygonal obstacles by computing the configuration space with
Minkowski sums.  While theoretically interesting, often, explicitly computing
the configuration space is too computationally intensive.  Describe another
method for performing motion planning.

You MAY (but do not need to) answer this question by researching on the
internet. Cite the sources you were using and give an explanation in your own
words

\section*{Problem 3 (10 Points)}

Consider an arrangement $\A$ of six lines $\ell_1, \ell_2, \ldots, \ell_6$ and
let $f$ be an arbitrary vertex, edge, or face of $\A$. Then $f$ has an
associated sign vector $(s_1, s_2, s_3, s_4, s_5, s_6)$, where for each $1 \le i
\le 6$:

$$
    s_i =
    \begin{cases}
        +1, & \text{if $f$ lies above $l_i$} \\
        0,  & \text{if $f$ lies on $l_i$} \\
        -1, & \text{if $f$ lies below $l_i$}
    \end{cases}
$$

\begin{enumerate}

    \item For each of the sign vectors below, give an arrangement of six lines
        that has a vertex, edge, or face with this sign vector. Label the lines
        and mark the vertex, edge, or face. Make the arrangement simple, if
        possible, or argue why the arrangement cannot be simple.
        \begin{enumerate}
            \item $(+1, +1, +1, +1, +1, +1)$
            \item $(+1, 0, 0, -1, -1, -1)$
            \item $(-1, 0, 0, -1, +1, -1)$
            \item $(+1, -1, -1, -1, -1, -1)$
        \end{enumerate}

    \item Can one find a single arrangement of lines that contains a vertex,
        edge, or face for each of the four sign vectors? If you can, provide an
        example.  If you cannot, argue why not.

\end{enumerate}


\section*{Problem 4 (10 Points)}

We saw in class a few applications of arrangements for solving geometric
problems. Find a problem whose solution was not described in class.  Give a
description of the problem and how one can use arrangements to solve the
problem.

You MAY (but do not need to) answer these questions by researching on the
internet. Cite the sources you were using and give an explanation in your own
words

\section*{Tips and Acknowledgements}

{\bf Acknowledgements:} Homework problems adapted from assignments of Carola
Wenk.


\end{document}
